\documentclass[]{article}

%opening
\title{Origin of Infinity}
\author{Riley Kavanagh}
\date{}




%packages
\usepackage{amsmath,amsfonts,amssymb,amsthm}
\usepackage{braket}
\usepackage[margin=2.5cm]{geometry}
\usepackage{bbold}
\usepackage{braket}
\usepackage{graphicx}
\usepackage{caption}
\usepackage{refstyle}


%example of helpful commands you can write
\newcommand{\ehx}{\hat{e}_x}
\newcommand{\ehy}{\hat{e}_y}
\newcommand{\ehz}{\hat{e}_z}
\newcommand{\bd}{\textbf}


\begin{document}

\maketitle

\begin{abstract}
I am beginning with the origin of \textit{Infinity}. Though this is mainly a proof of concept for our project; this also is an interesting part, that I think can contribute a lot.

\end{abstract}

\section{\textit{Infinity}}
Though the earliest use of the mathematical concept of \textit{Infinity}, it was not widespread with its symbol $\infty$ until the early 17th century (citation needed). The modern idea, outside of academic work is seen primarily in the infinite number line:

insert-numberline-picture



\end{document}
